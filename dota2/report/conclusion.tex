\documentclass[result.tex]{subfiles}

\begin{document}

\section*{\centering Conclusion}

In this paper I have used data mining techniques to find patterns in the drafting phase of DOTA 2 in professional matches by looking at the team compositions. Association and cluster analysis was both used to answer different questions.

Association rules gave us some ideas as to what to pick against certain combinations of heroes and also what combinations are common with specific heroes.

To find broader patterns in the team compositions, specifically the metagame, I used cluster analysis that found common combinations of heroes and what heroes are contested in professional games at different events.

These results do however not answer the reasons behind the choices that are being made by professionals and are therefore not by themselves fully complete, but rather a step in a pipeline of analyses.

\subsection*{Future Work}

The pipeline that was hinted at may consist of searching for interesting hero combinations (this paper) and then use those results to further analyze the whys in the gameplay itself for instance.

To further improve the evaluation of the techniques used in the analyses is to make it possible for the community to rate it by an online interface. That way it would be possible to gain feedback from people with various degree of knowledge about the game to determine if the results are useful to the targeted demographic.

\end{document}
