\documentclass[result.tex]{subfiles}

\begin{document}

\section*{\centering Result}

\subsection*{Association Analysis}

\subsubsection*{When should I pick hero X?}

To keep the scope realistic I decided to choose two particular heroes, Anti-Mage and Clinkz, to analyze when it should be picked according to patterns in professional matches.

Anti-Mage is usually a niche pick that have specific strengths as his name suggests. A mobile hero that is extra strong against mages so he is usually picked at the end of the drafting phase. The matches analyzed are those where he is picked as a 4th or 5th pick, 153 out of 180 matches, in Valve events as it suggest that he was a counter pick to the opposing teams heroes. The analysis for Anti-Mage was done by select the opposing team composition of the 153 matches and run the FP-Growth algorithm on the matrix, 153x5, and table \ref{tab:ass_antimage} shows rules that were found.

\begin{table}[H]
  \centering
  \begin{tabular}{ | c | c | }
    \hline
    antecedent & consequent \\ \hline
    Razor & Rubick \\ \hline
    Ember Spirit & Dark Seer \\ \hline
    Sand King & Rubick \\ \hline
    Naga Siren & Dark Seer  \\
    \hline
  \end{tabular}
  \caption{Association rules by FP-Growth algorithm of 153 team compositions against 4th or 5th pick Anti-Mage. The parameters were set to minimum support of 5 and minimum confidence of 0.4.}
  \label{tab:ass_antimage}
\end{table}

Clinkz is also a niche pick that is very mobile because he can become invisible and has high single target damage. He his, as Anti-Mage, usually a 4th or 5th pick, 93 out of 102 matches, and so a similar analysis was made which can be seen in table \ref{tab:ass_clinkz}.

\begin{table}[H]
  \centering
  \begin{tabular}{ | c | c | }
    \hline
    antecedent & consequent \\ \hline
    Dark Seer & Juggernaut \\ \hline
    Gyrocopter & Earthshaker \\ \hline
    Tiny & Io \\
    \hline
  \end{tabular}
  \caption{Association rules by FP-Growth algorithm of 180 team compositions against 4th or 5th pick Clinkz. The parameters were set to minimum support of 5 and minimum confidence of 0.5.}
  \label{tab:ass_clinkz}
\end{table}

\subsubsection*{Which hero should I pick along X?}

Similar to above, I decided to choose two heroes, Io the Wisp and Magnus. Io is very good at adding mobility and durability to heroes. Io is most commonly picked early in the draft since it can be combined with many heroes depending on the opponent's picks. Magnus is good at buffing melee heroes and initiate engagements

Here I used all the team compositions from 3028 matches, i.e. 6056 team compositions, in which Io occurs 451 times and magnus 165. The rules selected contained either Io or Magnus in either the antecedent or consequent. Table \ref{tab:ass_io} shows the rules found for Io by FP-Growth

\begin{table}[H]
  \centering
  \begin{tabular}{ | c | c | }
    \hline
    Antecedent & Consequent \\ \hline
    Earthshaker, Tiny & Io \\ \hline
    Tiny, Beastmater & Io \\ \hline
    Tiny, Rubick & Io \\ \hline
    Earthshaker, Io & Tiny  \\ \hline
    Tiny, Queen of Pain & Io \\ \hline
    Tiny, Batrider & Io \\
    \hline
  \end{tabular}
  \caption{Association rules containing Io by FP-Growth algorithm of 6028 team compositions. The parameters were set to minimum support of 10 and minimum confidence of 0.5.}
  \label{tab:ass_io}
\end{table}

and table \ref{tab:ass_magnus} contains the rules that include Magnus.

\begin{table}[H]
  \centering
  \begin{tabular}{ | c | c | }
    \hline
    Antecedent & Consequent \\ \hline
    Vengeful Spirit, Magnus & Juggernaut \\ \hline
    Templar Assassin, Magnus & Juggernaut \\ \hline
    Silencer, Magnus & Juggernaut \\ \hline
    Witch Doctor, Magnus & Juggernaut  \\
    \hline
  \end{tabular}
  \caption{Association rules containing Magnus by FP-Growth algorithm of 6028 team compositions. The parameters were set to minimum support of 5 and minimum confidence of 0.5.}
  \label{tab:ass_magnus}
\end{table}

\subsection*{Cluster Analysis}

In this section I will present the two different cluster analysis that have been conducted. The presented results are manually selected and only partial. The complete set of clusters can be found in the appendix at the end of the report. The analyses have utilized \textit{k}-modes and ROCK algorithms and follow the questions in the introduction, \textit{What are common team compositions overall?} and \textit{Do team compositions change between tournaments?}, and they are described below.

As a note in reading the tables, the numbers beside the hero names in the tables are the frequency of the corresponding hero in the cluster.

\subsubsection*{All Valve Events}

In this analysis I have clustered all team compositions from Valve events since 2012, consisting of 6056 observations, to find common team compositions or hero combinations over a longer period of time. Due to performance issues with ROCK on larger datasets I have only used \textit{k}-modes.

\begin{table}[H]
  \centering
  \begin{tabular}{ | c | p{12.5cm} | }
    \hline
    \multicolumn{2}{ | c | }{Clusters} \\
    \hline
    Size & Samples \\ \hline
    \multirow{2}{*}{614}
    & Queen of Pain: 195, Disruptor: 193, Dark Seer: 191, Lifestealer: 93, Rubick: 122 \\
    & Earthshaker: 43, Queen of Pain: 195, Gyrocopter: 61, Disruptor: 193, Dark Seer: 191 \\
    \hline
    \multirow{2}{*}{486}
    & Io: 203, Earthshaker: 82, Templar Assassin: 97, Witch Doctor: 176, Tiny: 80 \\
    & Io: 203, Bristleback: 23, Batrider: 48, Tiny: 80, Witch Doctor: 176 \\
    \hline
    \multirow{2}{*}{91}
    & Tinker: 14, Disruptor: 39, Lifestealer: 33, Slardar: 44, Clockwerk: 37 \\
    & Templar Assassin: 6, Disruptor: 39, Ogre Magi: 6, Lifestealer: 33, Slardar: 44 \\
    \hline
  \end{tabular}
  \caption{}
  \label{cl_all_kmodes}
\end{table}

\begin{table}[H]
  \centering
  \begin{tabular}{ | c | p{12.5cm} | }
    \hline
    \multicolumn{2}{ | c | }{Clusters} \\
    \hline
    Size & Samples \\ \hline
    \multirow{2}{*}{691}
    & Disruptor: 229, Dark Seer: 211, Lifestealer: 107, Queen of Pain: 219, Rubick: 126 \\
    & Disruptor: 229, Dark Seer: 211, Gyrocopter: 64, Queen of Pain: 219, Earthshaker: 66 \\
    \hline
    \multirow{2}{*}{197}
    & Sand King: 36, Razor: 27, Shadow Demon: 73, Mirana: 69, Juggernaut: 65 \\
    & Ogre Magi: 27, Shadow Demon: 73, Sand King: 36, Mirana: 69, Luna: 29 \\
    \hline
    \multirow{2}{*}{119}
    & Slark: 26, Invoker: 43, Beastmaster: 40, Rubick: 19, Winter Wyvern: 19 \\
    & Invoker: 43, Clinkz: 4, Earth Spirit: 6, Beastmaster: 40, Oracle: 15 \\
    \hline
  \end{tabular}
  \caption{}
  \label{cl_all_kmodes_mod}
\end{table}

\subsubsection*{Shanghai vs. Manila Majors 2016}

In order to investigate if team compositions changes between tournaments I chose the Shanghai Major, 598 observations, and the Manila Major, 642 observations, that was played roughly 3 months apart in 2016 (2016-02-25 to 2016-03-06 and 2016-06-03 to 2016-06-12 respectively).

\subsubsection*{The Manila Major 2016}

\begin{table}[H]
  \centering
  \begin{tabular}{ | c | p{12.5cm} | }
    \hline
    \multicolumn{2}{ | c | }{Clusters} \\
    \hline
    Size & Samples \\ \hline
    \multirow{2}{*}{21}
    & Queen of Pain: 11, Dark Seer: 12, Lifestealer: 9, Doom: 10, Lion: 9 \\
    & Earth Spirit: 6, Dark Seer: 12, Queen of Pain: 11, Lifestealer: 9, Lion: 9 \\
    \hline
    \multirow{2}{*}{13}
    & Dark Seer: 5, Gyrocopter: 3, Vengeful Spirit: 13, Doom: 11, Queen of Pain: 2 \\
    & Vengeful Spirit: 13, Dark Seer: 5, Puck: 4, Doom: 11, Lifestealer: 1 \\
    \hline
    \multirow{2}{*}{34}
    & Puck: 10, Juggernaut: 19, Faceless Void: 12, Enchantress: 19, Lion: 20 \\
    & Slardar: 5, Invoker: 8, Juggernaut: 19, Enchantress: 19, Lion: 20 \\
    \hline
  \end{tabular}
  \caption{3 out of 150 clusters found by ROCK with a threshold of 0.6 for cluster merging on observations from the Manila Major.}
  \label{tab:cl_manila_rock}
\end{table}

\subsubsection*{The Shanghai Major 2016}

\begin{table}[H]
  \centering
  \begin{tabular}{ | c | p{12.5cm} | }
    \hline
    \multicolumn{2}{ | c | }{Clusters} \\
    \hline
    Size & Samples \\ \hline
    \multirow{2}{*}{29}
    & Io: 21, Beastmaster: 13, Tiny: 17, Witch Doctor: 9, Queen of Pain: 3 \\
    & Io: 21, Dark Seer: 9, Sven: 7, Tiny: 17, Witch Doctor: 9 \\
    \hline
    \multirow{2}{*}{27}
    & Invoker: 6, Gyrocopter: 23, Oracle: 4, Dark Seer: 16, Rubick: 9 \\
    & Invoker: 6, Gyrocopter: 23, Oracle: 4, Dark Seer: 16, Bane: 8 \\
    \hline
    \multirow{2}{*}{23}
    & Invoker: 14, Faceless Void: 13, Enchantress: 8, Witch Doctor: 11, Spectre: 4 \\
    & Earth Spirit: 5, Faceless Void: 13, Invoker: 14, Witch Doctor: 11, Ember Spirit: 2 \\
    \hline
  \end{tabular}
  \caption{3 out of 150 clusters found by ROCK with a threshold of 0.6 for cluster merging on observations from the Shanghai Major.}
  \label{tab:cl_shanghai_rock}
\end{table}

\end{document}
