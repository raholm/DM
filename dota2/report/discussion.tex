\documentclass[result.tex]{subfiles}

\begin{document}

\section*{\centering Discussion}

In this section I will discuss the results that were presented previously. Starting off with the association analysis followed by cluster analysis.

\subsection*{Association Analysis}

\subsubsection*{When should I pick hero X?}

As mentioned before, Anti-Mage is a mobile hero that is durable against magic damage, and examining the rules found by FP-Growth in table \ref{tab:ass_antimage} all the heroes are either heavy magic damage dealers or excel at split pushing, i.e., be all over the map. That suggests that the rules are actually useful in finding combinations of heroes that a player should pick Anti-Mage against.

And for Clinkz, the results also make sense (table \ref{tab:ass_clinkz}), since he is a physical dealer he works well against the likes of Juggernaut that is more suited against magic damage dealers. He also has the surprise factor with invisibility to easily kill vulnerable heroes such as Io.

\subsubsection*{Which hero(es) should I pick along X?}

Io is a hero that have many pairings that are commonly seen in games where Io is picked. In table \ref{tab:ass_io} we can see certain combinations that are common and it is clear that Tiny is one of those---it is a well known pairing in the DOTA community. We can see that it is combined with initiators such as Earthshaker, Beastmaster, and Batrider that enhances the team with what Io is lacking.

\subsection*{Cluster Analysis}

\subsubsection*{Hero Combinations}

There are certain hero combinations that standout among those that have been played in Valve events such as \{ Queen of Pain, Disruptor, Dark Seer \} and \{ Io, Tiny, Witch Doctor \}. Given knowledge of these heroes, it is clear that these combinations of heroes synergize well together, in particular \{ Io, Tiny \} is a well known pairing. We also see some less common pairings such as \{ Invoker, Beastmaster \} and \{ Lifestealer, Slardar \} that synergize very well together.

\subsubsection*{Evolution of the Metagame}

From the selected clusters it is obvious that different heroes were prioritized between the two events, check the appendix for more results. We can see that \{ Io, Tiny \}, \{ Gyrocopter, Dark Seer \}, and \{ Invoker, Faceless Void, Witch Doctor \} were fairly common combinations of heroes at the Shanghai major while at Manila others were popular such as \{ Queen of Pain, Dark Seer, Lifestealer, Lion \}, \{ Doom, Vengeful Spirit \}, and \{ Lion, Juggernaut, Enchantress \}. From these results alone I can only speculate what that actually means in terms of gameplay, but since the events did occur fairly up close to each other I would perhaps be naive to assume a complete shift in the metagame.


\subsection*{Summary}

Overall, it seems that association rules can find combinations of heroes that are either good together or that specific heroes are good against. The parameters can of course be tweaked further for generating even more rules that may have been missed but are useful.

Clustering do seem to find useful hero combinations and can detect differences, but it does not tell the whole story alone because DOTA 2 is more than just hero picks. The same can be said above the association rules. The choice of number of clusters was arbitrarily set based on experimenting with different values and something that is important to consider for future studies.

\end{document}
