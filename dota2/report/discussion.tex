\documentclass[result.tex]{subfiles}

\begin{document}

\section*{\centering Discussion}

In this section I will discuss the results that were presented previously. Starting off with the association analysis followed by cluster analysis.

\subsection*{Association Analysis}

\subsubsection*{When should I pick hero X?}

As mentioned before, Anti-Mage is a mobile hero that is durable against magic damage, and examining the rules found by FP-Growth in table \ref{tab:ass_antimage} all the heroes are either heavy magic damage dealers or excel at split pushing, i.e., be all over the map. That suggests that the rules are actually useful in finding combinations of heroes that a player should pick Anti-Mage against.

And for Clinkz, the results also make sense (table \ref{tab:ass_clinkz}), since he is a physical dealer he works well against the likes of Juggernaut that is more suited against magic damage dealers. He also has the surprise factor with invisibility to easily kill vulnerable heroes such as Io.

\subsubsection*{Which hero(es) should I pick along X?}

Io is a hero that have many pairings that are commonly seen in games where Io is picked. In table \ref{tab:ass_io} we can see certain combinations that are common and it is clear that Tiny is one of those---it is a well known pairing in the DOTA community. We can see that it is combined with initiators such as Earthshaker, Beastmaster, and Batrider that enhances the team with what Io is lacking.

Overall, it seems that association rules can find combinations of heroes that are either good together or that specific heroes are good against.

\subsection*{Cluster Analysis}

\subsubsection*{Hero Combinations}

There are certain hero combinations that standout among those that have been played in Valve events such as \{ Queen of Pain, Disruptor, Dark Seer \}, \{ Io, Tiny, Witch Doctor \}, and  \{ Disruptor, Lifestealer, Slardar \}. Given knowledge of these heroes, it is clear that these combinations of heroes synergize well together, in particular \{ Io, Tiny \} is a well known pairing.

\subsubsection*{Evolution of the Metagame}

From the selected clusters it is obvious that different heroes were prioritized between the two events, check the appendix for more results. We can see that \{ Io, Tiny \}, \{ Gyrocopter, Dark Seer \}, and \{ Invoker, Faceless Void, Witch Doctor \} were fairly common combinations of heroes in the Shanghai major while in Manila other combinations were popular such as \{ Queen of Pain, Dark seer, Lifestealer, Lion \}, \{ Doom, Vengeful Spirit \}, and \{ Lion, Juggernaut, Enchantress \}. From these results alone you could perhaps guess that in Shanghai Major the metagame was a little more towards team fight heavy compositions compared to the Manila Major that was more towards early game skirmishes and pushes. However, the median game lengths only differed by 1 minute, 37 vs 36 minutes, and so did the mean duration, 40 vs 39 minutes, making it difficult to determine if the analysis has any validity. However, since the events did occur fairly up close to each other one would perhaps be naive to assume a complete shift in the metagame.

\end{document}
